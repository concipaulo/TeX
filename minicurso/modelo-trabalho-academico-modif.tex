% abtex2-modelo-trabalho-academico.tex, v-1.9.6 laurocesar
%% Copyright 2012-2016 by abnTeX2 group at http://www.abntex.net.br/
%%
%% This work may be distributed and/or modified under the
%% conditions of the LaTeX Project Public License, either version 1.3
%% of this license or (at your option) any later version.
%% The latest version of this license is in
%%   http://www.latex-project.org/lppl.txt
%% and version 1.3 or later is part of all distributions of LaTeX
%% version 2005/12/01 or later.
%%
%% This work has the LPPL maintenance status `maintained'.
%%
%% The Current Maintainer of this work is the abnTeX2 team, led
%% by Lauro César Araujo. Further information are available on
%% http://www.abntex.net.br/
%%
%% This work consists of the files abntex2-modelo-trabalho-academico.tex,
%% abntex2-modelo-include-comandos and abntex2-modelo-references.bib
%%

% ------------------------------------------------------------------------
% ------------------------------------------------------------------------
% abnTeX2: Modelo de Trabalho Academico (tese de doutorado, dissertacao de
% mestrado e trabalhos monograficos em geral) em conformidade com
% ABNT NBR 14724:2011: Informacao e documentacao - Trabalhos academicos -
% Apresentacao
% ------------------------------------------------------------------------
% ------------------------------------------------------------------------

\documentclass[
	% -- opções da classe memoir --
	12pt,				% tamanho da fonte
	% openright,			% capítulos começam em pág ímpar (insere página vazia caso preciso)
    openany,
	% twoside,			% para impressão em recto e verso. Oposto a oneside
	oneside,
	a4paper,			% tamanho do papel.
	% -- opções da classe abntex2 --
	chapter=TITLE,		% títulos de capítulos convertidos em letras maiúsculas
	%section=TITLE,		% títulos de seções convertidos em letras maiúsculas
	%subsection=TITLE,	% títulos de subseções convertidos em letras maiúsculas
	%subsubsection=TITLE,% títulos de subsubseções convertidos em letras maiúsculas
	% -- opções do pacote babel --
	english,			% idioma adicional para hifenização
	% french,				% idioma adicional para hifenização
	% spanish,			% idioma adicional para hifenização
	brazil				% o último idioma é o principal do documento
	]{abntex2}

% ---
% Pacotes básicos
% ---
\usepackage{lmodern}			% Usa a fonte Latin Modern
\usepackage[T1]{fontenc}		% Selecao de codigos de fonte.
\usepackage[utf8]{inputenc}		% Codificacao do documento (conversão automática dos acentos)
\usepackage{lastpage}			% Usado pela Ficha catalográfica
\usepackage{indentfirst}		% Indenta o primeiro parágrafo de cada seção.
\usepackage{color}				% Controle das cores
\usepackage{graphicx}			% Inclusão de gráficos
\usepackage{microtype} 			% para melhorias de justificação
% ---

% ---
% Pacotes adicionais, usados apenas no âmbito do Modelo Canônico do abnteX2
% ---
\usepackage{lipsum}				% para geração de dummy text
% ---

% ---
% Pacotes de citações
% ---
\usepackage[brazilian,hyperpageref]{backref}	 % Paginas com as citações na bibl
\usepackage[alf]{abntex2cite}	% Citações padrão ABNT

% ---
% CONFIGURAÇÕES DE PACOTES
% ---

% ---
% Configurações do pacote backref
% Usado sem a opção hyperpageref de backref
\renewcommand{\backrefpagesname}{Citado na(s) página(s):~}
% Texto padrão antes do número das páginas
\renewcommand{\backref}{}
% Define os textos da citação
\renewcommand*{\backrefalt}[4]{
	\ifcase #1 %
		Nenhuma citação no texto.%
	\or
		Citado na página #2.%
	\else
		Citado #1 vezes nas páginas #2.%
	\fi}%
% ---
\renewcommand{\rmdefault}{phv}
\renewcommand{\sfdefault}{phv}
\renewcommand{\ABNTEXchapterfontsize}{\normalsize\bfseries}	%  | Configura a fonte do título de capítulo
\renewcommand{\ABNTEXsectionfontsize}{\normalsize\bfseries}		      %  | Configura a fonte do título de seção
\renewcommand{\ABNTEXsubsectionfontsize}{\normalsize}
%% ---
\renewcommand{\imprimircapa}{%
  \begin{capa}%
	  \center                                                                     %  | Centraliza os elementos da capa
	  \vspace*{-48pt}                                                             %  | Espaço entre a margem superior da folha e a instituição
	  \ABNTEXchapterfont\bfseries\normalsize\imprimirinstituicao                  %  | Imprime o nome da instituição
	  \vfill                                                                      %  | Separa igualmente os elementos da capa
	  \ABNTEXchapterfont\bfseries\normalsize\MakeTextUppercase\imprimirautor      %  | Imprime o nome do autor
	  \vfill                                                                      %  | Separa igualmente os elementos da capa
	  \begin{center}
		  \ABNTEXchapterfont\bfseries\normalsize\MakeTextUppercase\imprimirtitulo                     %  | Imprime o título
		  % \vfill
          \\
		  \ABNTEXchapterfont\bfseries\normalsize\MakeTextUppercase\imprimirtipotrabalho               %  | Imprime o tipo de trabalho
	  \end{center}
	  \vfill                                                                      %  | Separa igualmente os elementos da capa
	  \textbf\imprimirlocal\\                                                     %  | Imprime o local
    \textbf\imprimirdata                                                        %  | Imprime a data
	\end{capa}}
% Informações de dados para CAPA e FOLHA DE ROSTO
% ---
\titulo{Relatório}
\autor{João\\ Maria\\ José}
\instituicao{
  UNIVERSIDADE TECNOLÓGICA FEDERAL DO PARANÁ\\
  COORDENAÇÃO DE ENGENHARIA CIVIL}
\tipotrabalho{Ponte forth}
\local{PATO BRANCO}
\data{2017}
% ---
\setlength\afterchapskip{18pt} 			            %  | Espaçamento após capítulo
\setlength\beforesecskip{18pt}             	%  | Espaçamento entre texto e seção
\setlength\aftersecskip{18pt}			              %  | Espaçamento entre seção e texto
\setlength\beforesubsecskip{18pt}               %  | espaçamento entre texto e subseção
\setlength\aftersubsecskip{18pt} 		            %  | Espaçamento entre subseção e texto
\setlength{\cftbeforechapterskip}{0pt plus 0pt} %  | Remover espaço antes de capítulo
\renewcommand*{\insertchapterspace}{}           %  | Configura o espaço do capítulo para zero
\makepagestyle{meuestilo}
  \makeevenhead{meuestilo}{\ABNTEXfontereduzida\thepage}{}{}  %  | Cabeçalho das páginas pares
  \makeoddhead{meuestilo}{}{}{\ABNTEXfontereduzida\thepage}   %  | Cabeçalho das páginas impares ou com um único lado
  \makeheadrule{meuestilo}{\textwidth}{0pt}                   %  | Linha no cabeçalho com espessura zero
  \makeevenfoot{meuestilo}{}{}{}                              %  | Rodapé das páginas pares
  \makeoddfoot{meuestilo}{}{}{}                               %  | Rodapé das páginas impares ou com um único lado
% ---
% Configurações de aparência do PDF final

% alterando o aspecto da cor azul
\definecolor{blue}{RGB}{41,5,195}

% informações do PDF
\makeatletter
\hypersetup{
     	%pagebackref=true,
		pdftitle={\@title},
		pdfauthor={\@author},
    	pdfsubject={\imprimirpreambulo},
	    pdfcreator={LaTeX with abnTeX2},
		pdfkeywords={abnt}{latex}{abntex}{abntex2}{trabalho acadêmico},
		colorlinks=true,       		% false: boxed links; true: colored links
    	linkcolor=black,          	% color of internal links
    	citecolor=black,        		% color of links to bibliography
    	filecolor=magenta,      		% color of file links
		urlcolor=blue,
		bookmarksdepth=4
}
\makeatother
% ---

% ---
% Espaçamentos entre linhas e parágrafos
% ---

% O tamanho do parágrafo é dado por:
\setlength{\parindent}{1cm}

% Controle do espaçamento entre um parágrafo e outro:
\setlength{\parskip}{0cm}  % tente também \onelineskip

% ---
% compila o indice
% ---
\makeindex
% ---

% ----
% Início do documento
% ----
\begin{document}

% Seleciona o idioma do documento (conforme pacotes do babel)
%\selectlanguage{english}
\selectlanguage{brazil}

% Retira espaço extra obsoleto entre as frases.
\frenchspacing

% ----------------------------------------------------------
% ELEMENTOS PRÉ-TEXTUAIS
% ----------------------------------------------------------
% \pretextual
% ---
% Capa
% ---
\imprimircapa
% ---
% ---
% inserir lista de ilustrações
% ---
\pdfbookmark[0]{\listfigurename}{lof}
\listoffigures*
\cleardoublepage
% % ---
%
% % ---
% % inserir lista de tabelas
% % ---
\pdfbookmark[0]{\listtablename}{lot}
\listoftables*
\cleardoublepage
% ---
% ---
% inserir o sumario
% ---
\pdfbookmark[0]{\contentsname}{meuestilo}
\tableofcontents*
\cleardoublepage
% ---
% ---
% inserir lista de abreviaturas e siglas
% ---
% \begin{siglas}
%   \item[ABNT] Associação Brasileira de Normas Técnicas
%   \item[abnTeX] ABsurdas Normas para TeX
% \end{siglas}
% ---

% ---
% inserir lista de símbolos
% ---
% \begin{simbolos}
%   \item[$ \Gamma $] Letra grega Gama
%   \item[$ \Lambda $] Lambda
%   \item[$ \zeta $] Letra grega minúscula zeta
%   \item[$ \in $] Pertence
% \end{simbolos}
% ---
% ----------------------------------------------------------
% ELEMENTOS TEXTUAIS
% ----------------------------------------------------------
\textual
\pagestyle{meuestilo}               %  | Aplica o estilo "meuestilo" ao texto (Nota 3)
\aliaspagestyle{chapter}{meuestilo} %  | Aplica o estilo "meuestilo" as páginas que contém título de capítulo
% ----------------------------------------------------------
% Introdução (exemplo de capítulo sem numeração, mas presente no Sumário)
% ----------------------------------------------------------
\chapter[Introdução]{Introdução}
% \addcontentsline{toc}{chapter}{Introdução}
% ----------------------------------------------------------

% ----------------------------------------------------------
% ---
% Capitulo com exemplos de comandos inseridos de arquivo externo
% ---
% \include{abntex2-modelo-include-comandos}
% ---
% ----------------------------------------------------------
% PARTE
% ----------------------------------------------------------

% ---

% ----------------------------------------------------------
% Finaliza a parte no bookmark do PDF
% para que se inicie o bookmark na raiz
% e adiciona espaço de parte no Sumário
% ----------------------------------------------------------
\phantompart

% ---
% Conclusão
% ---
% \chapter{Conclusão}
% ---

% ----------------------------------------------------------
% ELEMENTOS PÓS-TEXTUAIS
% ----------------------------------------------------------
\postextual
% ----------------------------------------------------------

% ----------------------------------------------------------
% Referências bibliográficas
% ----------------------------------------------------------
\bibliography{abntex2-modelo-references}

% ----------------------------------------------------------
% Glossário
% ----------------------------------------------------------
%
% Consulte o manual da classe abntex2 para orientações sobre o glossário.
%
%\glossary

% ----------------------------------------------------------
% Apêndices
% ----------------------------------------------------------

% ---
% Inicia os apêndices
% ---
% \begin{apendicesenv}
%
% % Imprime uma página indicando o início dos apêndices
% % \partapendices
%
% % ----------------------------------------------------------
% \chapter{Quisque libero justo}
% % ----------------------------------------------------------
%
% \lipsum[50]
%
% % ----------------------------------------------------------
% \chapter{Nullam elementum urna vel imperdiet sodales elit ipsum pharetra ligula
% ac pretium ante justo a nulla curabitur tristique arcu eu metus}
% % ----------------------------------------------------------
% \lipsum[55-57]
%
% \end{apendicesenv}
% % ---
%
%
% % ----------------------------------------------------------
% % Anexos
% % ----------------------------------------------------------
%
% % ---
% % Inicia os anexos
% % ---
% \begin{anexosenv}
%
% % Imprime uma página indicando o início dos anexos
% % \partanexos
%
% % ---
% \chapter{Morbi ultrices rutrum lorem.}
% % ---
% \lipsum[30]
%
% % ---
% \chapter{Cras non urna sed feugiat cum sociis natoque penatibus et magnis dis
% parturient montes nascetur ridiculus mus}
% % ---
%
% \lipsum[31]
%
% % ---
% \chapter{Fusce facilisis lacinia dui}
% % ---
%
% \lipsum[32]
%
% \end{anexosenv}

%---------------------------------------------------------------------
% INDICE REMISSIVO
%---------------------------------------------------------------------
% \phantompart
% \printindex
%---------------------------------------------------------------------

\end{document}
