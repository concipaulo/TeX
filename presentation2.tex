\documentclass[14pt, aspectratio=169]{beamer}
%
\setbeamertemplate{navigation symbols}{}
% \usepackage{beamerthemeshadow}
\usepackage[utf8]{inputenc}
\usepackage[portuguese]{babel}
\usepackage{graphicx}
\usetheme[progressbar=frametitle]{metropolis}
% \usetheme[]{Warsaw}
\definecolor{cadmiumorange}{rgb}{0.93, 0.53, 0.18}
\setbeamertemplate{frame numbering}[fraction]
\useoutertheme{metropolis}
\useinnertheme{metropolis}
\usefonttheme{metropolis}
% \usecolortheme{spruce}
% \usecolortheme{beetle}
\usecolortheme{seahorse}
\setbeamercovered{transparent=5}
\setbeamertemplate{blocks}[rounded][shadow=false]

\addtobeamertemplate{block begin}{\setlength{\textwidth}{0.7\textwidth}}{}

\usepackage{xcolor}
\usepackage{amsmath}
\usepackage{multido}
\usepackage[most]{tcolorbox}
\usepackage[alf]{abntex2cite}
\usepackage{ragged2e}
\usepackage{hyperref}
\hypersetup{pdfpagemode=FullScreen}

% Information
\title{\textit{Verificação e Validação Numérica do Algoritmo GMRES
  na Solução de Problemas de Transferência de Calor Utilizando Método de
  Diferenças Finitas}}
% \subtitle{Autor: Paulo Conci}
\author{Autor: Paulo G. Conci \\ Orientador: Francisco A. A. Gomes}
\institute{UTFPR}
\date{22 de Novembro de 2017}
\setbeamercolor{title}{parent=structure,fg=gray,bg=yellow}
% \setbeamercolor{subtitle}{parent=structure,fg=cyan,bg=yellow}
% \setbeamercolor{author}{parent=structure,fg=cyan,bg=yellow}

%===================================================================
\begin{document}

% Title page
\begin{frame}[plain]
  \maketitle
\end{frame}

\begin{frame}[plain]{Organização}
    \tableofcontents
\end{frame} 

\section{Introdução}
\begin{frame}{Introdução}
\begin{itemize}
\item Transferência de Calor \\[10pt]
\item Método de Diferenças Finitas (FDM) \\[10pt]
\item Sistemas Lineares Altamente Esparsos
\end{itemize}
\end{frame}

\section{Procedimento Computacional} 
\begin{frame}{Procedimento Computacional}
\begin{itemize}
    \item Fatoração LU Incompleta (ILU)\\[10pt]
    \item Relaxação sucessiva
        \begin{itemize}
        \item Gauss-Siedel
        \end{itemize}
    \item Subespaço de Krylov
        \begin{itemize}
        \item \textit{Generalized Minimal Residual} (GMRES)\\[10pt]
        \end{itemize}
\end{itemize}
\end{frame}

\section{Verificação e Validação (V\&V)}

\begin{frame}{Verificação}
    \justifying
     O processo de determinar que um modelo ou simulação implementada e
    seus dados associados representam de forma precisa a descrição e especificações do
    desenvolvedor~\cite{dodi5000.61},~\cite{aiaa1998}.

    Verificação de código é distinta de verificação de solução e deve ser feita a
    \textit{priori}, mesmo que ambas utilizam procedimento de refino de
    malha~\cite{asme2009}.
\end{frame}

\begin{frame}{Validação}
    \justifying
     O processo de determinar em que grau o modelo, simulação e seus dados
     associados são uma representação precisa do mundo real sob a perspectiva
     dos fenômenos a serem
     estudados~\cite{aiaa1998},~\cite{dodi5000.61},~\cite{nasa7009a}.
\end{frame}

\begin{frame}{Verificação e Validação (V\&V)}
    \begin{itemize}
         \item Verificação: Estamos resolvendo as equações de forma correta? 
         \item Validação: Estamos resolvendo as equações certas?
    \end{itemize}
\end{frame}

\section{Resultado Esperado}

\begin{frame}{Resultado Esperado}
    \justifying
    Tem-se como objetivo esperado a obtenção de um código capaz de resolver
    problemas de transferência de calor, tanto estacionário quanto transiente,
    respeitando os conceitos de verificação e validação apresentados. 
\end{frame}

\begin{frame}[allowframebreaks]{Referências}
    \bibliography{bibfile.bib}
\end{frame}

\begin{frame}[standout]
Dúvidas?
\end{frame}

\begin{frame}[standout]
Obrigado!
\end{frame}

\end{document}
